\section{Calendar}

The calendar is a new functionality in SalesPoint 2011 to manage appointments.

\subsection{\code{Calendar} - The central menagement for appointments}
The calendar is an interface that provides simple functionality to store calendar entries persistently in and recover them from the underlaying database.
With the \code{PersistentCalendar}-class there exists an implementation of the calendar interface that can be worked with. This class provides methods
to add and remove entries and filter them by different criterias.  
The \code{PersistentCalendar} calendar itself has no identifier or other attributes, it's just a container that provides methods to easy manage all calendar entries.
So there is no need to keep a reference to any calendar object, every time one is needed, it can be created new.


\subsection{\code{CalendarEntry} - Keeps information about a single appointment}

With \code{CalendarEntry} there exists an interface to set, store and access information about a single appointment, which is implemented
by \code{PersistentCalendarEntry}.
Every \code{PersistentCalendarEntry} must have an owner, title, start and end date.
Typically the owner is the creator of the entry and also gets all capabilities for this entry.
Capabilities are
\begin{itemize}
    \item \code{OWNER} - indicates the owner of an entry who is typically also the creator of it
    \item \code{READ} - indicates who can read this entry
    \item \code{WRITE} - indicates who can change this entry
    \item \code{DELETE} - indicates who can delete the entry
    \item \code{SHARE} - indicates who can share the entry to other users
\end{itemize}.
All of these capabilities but \code{OWNER}, can be given to and removed from a user in relation to this specific entry.
Because a calendar entry does not check if a user has the capability to do something with it, the programmer has to check the capabilities before manipulating the entry.
Besides the minimum information a calendar entry can also have a description which provides more information about it and a repeat count and repeat step which say how 
often the entry is beeing repeated and how long the timespan between two repetitions is. 