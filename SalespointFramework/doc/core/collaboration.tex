\chapter{Collaboration}

The \salespoint{} framework is much more than a pure collection of classes and interfaces. It provides also processes between packages that are triggered automatic and facilitate consistent employment. This chapter gives attention to that dependencies between packages and describes how they collaborate.  

Figure \ref{package_overview} illustrates the most important dependencies between the \salespoint{} packages. As you can see nearly all packages depending among each other.\\ 

The central class that connects all features in \salespoint{} is the \code{Shop}. The most packages access this class to communicate with other packages. Therefore the \code{Shop} contains all interfaces which are global connected. This class should also be the first accesspoint for software engineers to request the seperate regions of \salespoint{}.\\ 

Another package that collaborate with nearly all packages is the \code{Order} package. \code{OrderLines} are using the interfaces from \code{Product} package to identify product instances and calculate their prices. \code{Orders} also communicate with the \code{UserManager}, to get information about involved users.\par 
When the \code{Order} will be completed, it communicates with the \code{Inventory} (via \code{Shop} class) and removes the considered product instances from that \code{Inventory}. \par
Before completion, the \code{Order} have to be payed. An \code{Order} which has changed to status \code{PAYED} will automatic access the accountancy and create a corresponding \code{AccountancyEntry} which represents the payment.\\ 




