\chapter{Preface}

\section{Typographic Conventions}
Two typographic conventions are employed throughout this document to highlight specific phrases.
The following paragraphs describe when and why these highlightings are used:
\\

\code{Mono-spaced Blue}
\\

The mono-spaced, blue font is used to denote variable names, class names, type names, java keywords, java package names, and so forth.
\\

\textit{Proportional Italic}
\\

Proper names and termini are printed in proportional, italic font.
\\

\section{Introduction}
The Salespoint Framework is intended to minimize developing effort of point-of-sale applications.
Salespoint 2010 users complained about complexity, missing features and bugs.
Thus, the decision was made to re-design and re-implement the framework from scratch.
Our development goal was an easy-to-use framework primarily targeted for educational purposes.
As such, \salespoint{} is not taylored to any specific application, but designed with a wide area of applications in mind.

Models and design patterns employed in \salespoint{} are inspired by ``Enterprise Patterns and MDA: Building Better Software with Archetype Patterns and UML'' by Jim Arlow~\cite{MDA}.
An overview of the functionality of and new features in \salespoint{} are detailed in this document.

We would like to thank all Salespoint users who submitted their feedback and encourage future users of \salespoint{} to do the same.
